%************************************************
\chapter{Functions}\label{ch:functions}
%************************************************

The concept of creating and calling functions is a very powerful one. In fact,  it is a concept present in nearly all programming languages. The main idea behind this concept is to group lines of code that perform a particular task together, and allow them to be executed multiple times with a simple function call. By properly using functions, users can write scripts that are much easier to understand, as well as to avoid having to write the same lines of code in multiple places inside your script.

To create a function in \mycommand{bash}, the following syntax can be used:

\begin{command_line}
function FUNCTION_NAME {
    #line or lines of code that are executed every time the function is called
}
\end{command_line}
Where \mycommand{FUNCTION\_NAME} stands for the name the function was given. It is also possible to create a function using a different syntax as shown below:
\begin{command_line}
FUNCTION_NAME (){
    #line or lines of code that are executed every time the function is called
}
\end{command_line}
Both syntaxes are interpreted the same way by the shell, so the choice of usign one or another is a matter of personnal preference\marginnotes{This book's author, for instance, prefers the first one}. Note that a function can be called at any point in your script, after the lines in which it has been defined, by simply writing its name. 

To exemplify how powerful functions are, take a look at Listing \ref{fil:script_nofunction}. In it we have created a little calculator. It keeps asking the user for which operation to use, as well as for two numbers, before outputing the desired result. Note that we repeat the lines in which we ask for the numbers at four different points in the script.

\begin{source_code_float}{Bash}{Script for a calculator program.}{fil:script_nofunction}
#!/bin/bash
while(true)
do
    clear
    echo "Choose from the following operations:"
    echo "Addition [+], Subtraction [-], Multiplication [/]Division"
    read -p "Your choice: " choice
    if [ $choice == "a" ] ; then
        echo "Enter first integer: " 
        read int1
        echo "Enter second integer: "
        read int2
        res=$((int1+int2))
     elif [ $choice == "-" ] ; then
        echo "Enter first integer: "
        read int1
        echo "Enter second integer: " 
        read int2
        res=$((int1-int2))
     elif [ $choice == "*" ] ; then
        echo "Enter first integer: " 
        read int1
        echo "Enter second integer: " 
        read int2
        res=$((int1*int2))
    elif [ $choice == "/" ] ; then
        echo "Enter first integer: " 
        read int1
        echo "Enter second integer: " 
        read int2
        res=$((int1/int2))
    else
        res=0
        echo "wrong choice!"
    fi

    echo "The result is:  $res"
    echo "Do you wish to continue? [y]es or [n]o: "
    echo  ans
    if [ $ans == 'n' ]
        then
         echo "Exiting the script. Have a nice day!"
        break
    fi
done
\end{source_code_float}

We can avoid having to repeat the same lines of code multiple times by creating a function called \mycommand{get\_inputs}, and calling it from multiple points in our script. See Listing \ref{fil:script_function}.

\begin{source_code_float}{Bash}{Script for a calculator program using a function.}{fil:script_function}
#!/bin/bash

function get_inputs{
    echo "Enter first integer: " 
    read int1
    echo "Enter second integer: " 
    read int2
}

while(true)
do
    clear
    echo "Choose from the following operations:"
    echo "Addition [+], Subtraction [-], Multiplication [/]Division: "
    read -p "Your choice: " choice
    if [ $choice == "a" ] ; then
        get_inputs
        res=$((int1+int2))
     elif [ $choice == "-" ] ; then
        get_inputs
        res=$((int1-int2))
     elif [ $choice == "*" ] ; then
        get_inputs
        res=$((int1*int2))
    elif [ $choice == "/" ] ; then
        get_inputs
        res=$((int1/int2))
    else
        res=0
        echo "wrong choice!"
    fi

    echo "The result is:  $res"
    echo "Do you wish to continue? [y]es or [n]o: "
    echo  ans
    if [ $ans == 'n' ]
        then
         echo "Exiting the script. Have a nice day!"
        break
    fi
done
\end{source_code_float}
By calling the \mycommand{get\_inputs} function at line 17 of Listing \ref{fil:script_function}, we are in fact executing the code in lines 4 to 7, which asks for two inputs and save these inputs on variables \mycommand{int1} and \mycommand{int2}. The same process happens on lines 20, 23, and 26. Note that, before the function \mycommand{get\_inputs} is called for the first time, the code in lines 4 to 7 has not yet been executed. This means that the variables \mycommand{int1} and \mycommand{int2} are only going to be defined after the first function call. As a result, trying to access the contents of these variables before the first function call will result in empty values.

\section{Escope of Variables}
By default, all variables defined inside a function are available anywhere inside our script. I.e., we can use these variables in the main body of our script, or even inside other functions defined in the same script\marginnotes{As long as these functions are only called after the function that creates them}. In technical terms, these variables are \textbf{global} variables.

There are scenarios in which you may not want to have a variable created inside a function to have global escope. For example,  you might want to prevent this variable from overwritting the value of other variables with identical names defined elsewhere. For these scenarios, you can set a variable to have a local escope\marginnotes{I.e., it is only defined inside the function} by defining them using the \mycommand{local} keyword before assigning them a value. See the example in Listing \ref{fil:script_local}.

\begin{source_code_float}{Bash}{Script containing a local variable.}{fil:script_local}
#!/bin/bash
function get_name {
    echo "Enter your name (function): "
    local name
    read name
    echo "Hello $name" #displays name defined inside the function
}
echo "Enter your name (main): 
read name
get_name
echo "Hello $name" #displays name defined in main
\end{source_code_float}
Note that in Listing \ref{fil:script_local}, the \mycommand{echo} statement in line 14 will print the value for the variable \mycommand{name} as defined in the main body of the script (on line 12), not the value of the variable \mycommand{name} defined inside the function (on line 7). As a matter of fact, the local variable \mycommand{name} defined inside the function ceases to exist as soon as we exit the function.


\section{Passing arguments to Functions}
Functions can take arguments in the same manner as a script takes arguments. I.e.,  we can call a function together with one or more arguments, separated by spaces, as we discussed in Section \ref{sec:passing_arguments}. These arguments are accessed inside the function according to their position. I.e., the first argument can be accessed via \mycommand{\$1}, the second argument via \mycommand{\$2}, and so it goes. See the example in Listing \ref{fil:function_arguments}.

\begin{source_code_float}{Bash}{Script containing a function that takes arguments.}{fil:function_arguments}
#!/bin/bash
function print_message {
    echo "Hello $1 $2,  today is : " `date`
}
echo "Enter your first name: 
read first_name
echo "Enter your last name: "
read last_name
print_message $first_name $last_name 
\end{source_code_float}
Note that in Listing \ref{fil:function_arguments}, the values stored in variables \mycommand{first\_name} and \mycommand{last\_name} are accessed from inside the function using \mycommand{\$1} and \mycommand{\$2}, respectively. Also, note that the backticks (` `) that surround the \mycommand{date} command inside the function are necessary. They guarantee that the script will display the output of the \mycommand{date} command, as opposed to the string ``\textit{date}''.

It is important to mention that functions do not have direct access to arguments passed to the script during the script call. For example, as shown in Section \ref{sec:passing_arguments}, when calling a script with: \mycommand{./myscript \$arg1 \$arg2}, you can access the values stored in \mycommand{arg1} and \mycommand{arg2} from inside the main body of the script using \mycommand{\$1}, and \mycommand{\$2}. However, you cannot use \mycommand{\$1} and \mycommand{\$2} to directly access these values from within a function defined inside this script. In order to access these values from within a function, your script must redirect them as arguments during the function call. See the example in Listing \ref{fil:function_redirectedarguments}.
 
 \begin{source_code_float}{Bash}{Script function that takes redirected arguments.}{fil:function_redirectedarguments}
#!/bin/bash
function print_message {
    echo "Hello $1 $2,  today is : " `date`
}
print_message $1 $2 
\end{source_code_float}


\section{Returning an Exit Status from a Function}
Functions in \mycommand{bash} can return an exit status\marginnotes{Only integers between 0 and 255 can be used}. This can be used in logical expressions, for example, where an exit status of \mycommand{0} is considered a logical \mycommand{true}, and any other exit status is considered a logical \mycommand{false}. See the example in Listing \ref{fil:function_exitstatus}.

 \begin{source_code_float}{Bash}{Script containing a function that returns an exit status.}{fil:function_exitstatus}
#!/bin/bash
function right_credential {
    echo "enter your username" 
    read user
    if [ $user == "marcel" ] || [ $user == "john" ] ; then
        return 0
    else
        return 1
    fi
}
if right_credential ; then
    echo "You entered a valid username"
else
    echo "You did not enter a valid username"
fi
\end{source_code_float}
Note that, in Listing \ref{fil:function_exitstatus} we do not put the \mycommand{right\_credential} function call within square brackets, as we normally do with other logical expressions. This is due to the fact that the return value of the function is already either \mycommand{true} or \mycommand{false}. In our previous examples, we had expressions that needed to be evaluated as \mycommand{true} or \mycommand{false} such as \mycommand{[ \$var -leq 10 ]}, or the logical expressions in \ref{fil:script_function}.

It is important to note that the \mycommand{return} keyword in \mycommand{bash} is only used to return exit codes. This is quite different than what most other computer languages do. For example, in most computer languages you can directly assign the return value of a function to a variable with an statement such as: \mycommand{my\_var=my\_function}. However, in \mycommand{bash} to pass a value from a function directly to a variable, we need to echo it and evaluate the output of the echo. See Listing \ref{fil:function_exitstatus2}.

 \begin{source_code_float}{Bash}{Script showing how to save values echoed from functions into variables.}{fil:function_exitstatus2}
#!/bin/bash

function my_funct1 {
    return 3
}
function my_funct2 {
    echo 3
}
my_var1=$my_funct1 #wrong
echo "The value in my_var1 is: $my_var1"
my_var2=$(my_funct2) #correct
echo "The value in my_var2 is: $my_var2"
\end{source_code_float}

\section*{Exercises}
\addcontentsline{toc}{section}{Exercises}

\begin{exercises}
   For the exercises 1 to 4, you will need to refer to the script shown below:
\begin{source_code}
#!/bin/bash
function my_func1{
   local var1
   fvar1=10
   fvar2=20
}
function my_func2{
   local var3
   fvar3=30
   fvar4=15
}
mvar1=10
\end{source_code}
   \item Can you access the contents of \mycommand{fvar1} and \mycommand{fvar2} from within the function \mycommand{my\_func2}?
   \item Can you access the contents of \mycommand{fvar1} and \mycommand{fvar3} from within the main body of the script?
   \item Can you access the contents of \mycommand{fvar2} and \mycommand{fvar4} from within the main body of the script?
   \item Can you access the contents of \mycommand{mvar1} within the functions \mycommand{my\_func1} or \mycommand{my\_func2}?
   \item Create a function called \mycommand{is\_weekend} that asks what day of the week it is, saves the answer in a variable, and then returns a value of \mycommand{true} if the answer was \mycommand{Saturday} or \mycommand{Sunday}. Otherwise, it should return a value of false. Your function should be able to work properly with the script shown below:
\begin{source_code}
#!/bin/bash
function is_weekend{
    #write your function code here
}
if is_weekend ; then
   echo "Today is a weekend day"
else
   echo "Today is a work day"   
fi
\end{source_code}   
     
   \item The following script uses a lot of repeated lines of code. Rewrite this script using a function called \mycommand{greeting} in order to avoid repetition. \textit{Hint: This function might take an argument}.

\begin{source_code}
#!/bin/bash

echo "Provide your username: "
read username

if [ $username == "john" ] ; then
    echo "Hi john, welcome back to our system "
    echo "Today is" `date` 
    echo "Remember to save all your information before logging off"
elif [ $username == "paul" ] ; then
    echo "Hi paul, welcome back to our system "
    echo "Today is" `date`    
    echo "Remember to save all your information before logging off"
elif [ $username == "george" ] ; then
    echo "Hi george, welcome back to our system "
    echo "Today is" `date`    
    echo "Remember to save all your information before logging off"
elif [ $username == "ringo" ] ; then
    echo "Hi ringo, welcome back to our system "
    echo "Today is" `date`    
    echo "Remember to save all your information before logging off"
else
    echo "You have entered a wrong username"
fi
\end{source_code}
\end{exercises}