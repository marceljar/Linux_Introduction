%************************************************
\chapter{Package Management}\label{ch:packages}
%************************************************

An \acs{OS}’s main job is to provide users with an environment in which they can use tools, also known as applications or packages, in order to accomplish a variety of tasks.

In fact, the popularity of an \acs{OS} is closely tied to that of the applications it can run. For example, Microsoft Windows became the most popular desktop \acs{OS} in the early nineties partly due to the fact that it could be used to run applications belonging to the the Microsoft Office suite, such as MS Word, MS Excel, and MS Power Point.

With regards to Linux, many of its distributions have been very successful in the server market partly due to the availability of powerful server applications, such as Apache, the availability of database
management systems, such as MySQL, as well as the availability of a large number of software development and sysadmin tools.


We have so far, in this book, used a number of tools that come
by default in most Linux distributions, such as \mycommand{vim}, \mycommand{nano}, \mycommand{cat}, \mycommand{grep}, etc. For some other tools, such as \mycommand{tree}, which does not come installed by default in some systems, we provided a command to install them without explaining how it works\marginnotes{\mycommand{sudo apt-get install tree}}. In this chapter, we will bridge this gap. We will show how to install, update, and remove packages in Linux, and we will also introduce the concepts of dependencies and repositories.

\section{Comparison between Windows and Linux Software Management}

In Windows, the most commom way to install a new application is to follow these two steps:
\begin{enumerate}
\item Download its installer from the website of its developer.
\item Run the downloaded installer. This might require answering a few questions, such as turning on automatic updates, entering licensing information, etc.
\end{enumerate}

Note that, in Windows, each application needs to have its own installer. There is no centralized tool to handle installing new tools.

In most Linux distributions, on the other hand, installing new applications is a task performed by centralized systems. These centralized systems work with file packages and, therefore, are referred to as package management systems.

A Linux software package is a file that contains not only the software to be installed, but also meta information about the software, such as: its current version, any dependencies\marginnotes{Dependencies are covered later in this chapter}, as well as any scripts that are to run at the end of the installation process.

\section{Package Formats}
\label{sec:package_formats}

Package management can vary drastically depending on which Linux distribution is being used. Different distributions can use different package formats. The two most popular formats for Linux packages are:

\begin{description}
\item[.rpm] RPM Package Manager, is the package format employed by Linux distributions maintained by Red Hat, such as Red Hat Enterprise Linux, Fedora, and CentOS, as well as OpenSUSE.
\item[.deb] Debian Package, is the package format employed by Linux distributions based on Debian, such as: Debian itself, Ubuntu, Mint, Kali, etc.
\end{description}

In this chapter, we will focus on the package management using packages in the \mycommand{.deb} format. However, most concepts introduced here have a direct equivalent for \mycommand{rpm} packages. A comparison table is provided at the end of the chapter.

Note that most distributions also provide graphical tools to perform package management tasks, such as installing and removing packages. These tools are mostly self-explanatory and are not discussed in this book.

\section{Package Management}

\subsection{dpkg}

The package management tool for \mycommand{.deb} packages is called \mycommand{dpkg}. To use it to install a package whose \mycommand{.deb} file is located in the current working directory, the following command is used:

\begin{command_line}[make]
sudo dpkg -i PACKAGE_NAME.DEB
\end{command_line}

To remove a previously installed application, the following command can be used\marginnotes{Note that we use the application name, not the package file name in this case}:

\begin{command_line}[make]
sudo dpkg -r APLICATION_NAME
\end{command_line}

Note that, installing and removing applications directly using \mycommand{dpkg} is normally not recommended. The reason being that this command cannot handle dependencies properly.
\vspace{1cm}

\begin{my_box}[Dependencies]
\label{box:dependencies}

As mentioned in \ref{sec:unix}, the Unix philosophy dictates that software development should be modular. I.e., instead of creating complex monolithic applications, developers should write a number of simple applications and combine them together.

An interesting consequence from this philosophy is that, many times, these simple applications can be combined in different ways to achieve different goals. Also, these simple applications can be combined with other simple applications, to achieve even more complex goals.

Another consequence of this philosophy is that many applications are not self-contained. They need a number of other applications to be also installed in the same \acs{OS} in order to function. These required applications are called dependencies.

For example, in order to use \mycommand{shutter}, a tool to take screenshots of parts of the screen, more than thirty smaller applications are required.
\end{my_box}


\subsection{\mycommand{apt-get}}

The Debian Advanced Package Tool \acs{APT} is one of its most important features of the Debian distribution.

It provides a method to install packages in a way that guarantees that all dependencies are satisfied. Also, it introduces the concept of repositories\marginnotes{Repositories are discussed in XXX}, I.e., online lists of approved applications with links to its .deb package files.

In what follows, we present a number of methods to install, remove, and update packages using \mycommand{apt-get}, the command line tool supplied by \acs{APT}.

\subsubsection*{\mycommand{install}}
\label{sec:apt-get-install}

To install a new package from one of the currently linked repositories, the following command can be used:

\begin{command_line}[make]
sudo apt-get install PACKAGE #install the chosen package
\end{command_line}

Once this command is executed, the package management system will search for this package in the linked repositories. If such a package is found, the system will read the package information, build a dependency tree, and ask the user if it is OK to install the package together with all the required dependencies. See Listing \ref{fil:apt-get-install}.

After the user enters \mycommand{Y} to allow the installation to proceed, the system will download the package and any required dependencies from the repositories, extract their contents, and run any scripts that are indicated in the \mycommand{.deb} package file.  

\begin{command_line_float}{make}{Using apt-get to install \mycommand{curl}.}{fil:apt-get-install}
@@marcel@dell:~$sudo apt-get install curl@@ 
Reading package lists... Done
Building dependency tree       
Reading state information... Done
The following extra packages will be installed:
  libcurl3
The following NEW packages will be installed:
  curl
The following packages will be upgraded:
  libcurl3
1 upgraded, 1 newly installed, 0 to remove and 78 not upgraded.
Need to get 296 kB of archives.
After this operation, 315 kB of additional disk space will be used.
Do you want to continue? [Y/n]
\end{command_line_float}


\subsubsection*{\mycommand{remove}, \mycommand{purge}, and \mycommand{autoremove}}

To remove a previously installed package, the following command can be used:

\begin{command_line}[make]
sudo apt-get REMOVE PACKAGE #remove the chosen package
\end{command_line}

Once this command is executed, the package management system will search for this package within its current list of installed applications. If such a package is found, the system will read the package information, build a dependency tree, and ask the user if it is OK to remove the package. See Listing \ref{fil:apt-get-remove}.

\begin{command_line_float}{make}{Using apt-get to remove \mycommand{curl}.}{fil:apt-get-remove}
@@marcel@dell:~$sudo apt-get remove curl@@ 
Reading package lists... Done
Building dependency tree       
Reading state information... Done
The following packages will be REMOVED:
  curl
0 upgraded, 0 newly installed, 1 to remove and 78 not upgraded.
After this operation, 315 kB disk space will be freed.
Do you want to continue? [Y/n] 
\end{command_line_float}

Note that this command, by default, will not delete any configuration files that were set by this application during its installation process. In order to remove an application and delete all configuration files, the following command should be used:
\begin{command_line}[make]
sudo apt-get purge PACKAGE #remove package and config files 
\end{command_line}

Also, this command will not remove, by dafault, any package dependencies that were installed together with this package\marginnotes{Note that \mycommand{libcurl3} was installed together with curl in Listing \ref{fil:apt-get-install}, but not removed with curl in Listing \ref{fil:apt-get-remove}}. To remove dependencies that are no longer required\marginnotes{Because the applications that required them were removed}, the following command can be invoked.

\begin{command_line}[make]
sudo apt-get autoremove #remove orphaned dependencies 
\end{command_line}


\begin{my_box}[Repositories]
\label{box:repositories}

Most Linux distributions come with a number of applications installed out-of-the-box. Such applications normally include: browsers, office applications, calculators, notepads, etc.

Besides the applications installed by default, most Linux distributions also come with a number of online repositories. These repositories contain packages that can be easily installed and mantained using \acs{APT}. For example, in  Section \ref{sec:apt-get-install}, the package \mycommand{curl} could be installed using \mycommand{apt-get install curl} because it is present in a repository provided by default with Ubuntu.

A good analogy for Linux repositories is that they can be compared to the  Google Play and the Apple App Store. While Android phones and iPhones already come with a number of applications (called apps), more apps can be installed and maintained using Google Play (Android) or App Store (Apple).

You can see a list of repositories that are provided by default in Ubuntu by checking the file \mycommand{/etc/apt/source.list}. Some applications, like Dropbox or Google Chrome add a number of repositories to the system. This, way they can keep being updated via \acs{APT}, even without being present in the default Ubuntu respositories. The repositories added by applications can be seen at the \mycommand{/etc/apt/source.list.d} folder.

To add more repositories to your list, you can either manually edit \mycommand{/etc/apt/source.list}, or you can use the command \mycommand{sudo add-apt-repository ppa:REPO\_NAME}, where \mycommand{REPO\_NAME} stands for the name of the repository that you want to add. For example, to add a repository for Wine, a tool to run Windows applications in Linux, you can use \mycommand{sudo add-apt-repository ppa:ubuntu-wine/ppa}. Always be careful to only add trusted repositories. Otherwise, you might infect your system with malware.

The advantage of using the default repositories to install new applications, instead of using \mycommand{.deb} packages, is that those applications are consistently checked for bugs, inconsistencies, and malware. Therefore, as long as you trust the team who maintains your Linux Distribution, you can trust these applications. Plus, whenever an application receives an update, it is also updated in the repository. Hence, the user can use the \acs{APT} system to update them as well.
\end{my_box}

\subsubsection*{\mycommand{update} and \mycommand{upgrade}}

Besides taking care of installing and remolving packages, \acs{APT} also takes care of ensuring that the latest version of each package is properly installed in your system. For example, if a package goes from a version \mycommand{2.6} to a version \mycommand{2.7} or even \mycommand{3.0}, \acs{APT} can be used to update the index file of the package, and then upgrade the package with the contents in the new index. 

To synchronize the package index files, for all packages currently installed in your system, with the most up to date versions of these packages\marginnotes{According to the repositories specified in \mycommand{/etc/apt/sources.list} and \mycommand{/etc/apt/sources.list.d}}, the \mycommand{update} command shown below can be used. Note that this command will not upgrade any packages. It will simply update the package index files, allowing these packages to be updated using the \mycommand{upgrade} command.

\begin{command_line}[make]
sudo apt-get update #update the package index files
\end{command_line}

After running \mycommand{sudo apt-get update}, you can upgrade all packages to their latest version running the command below\marginnotes{Remmeber to always use \mycommand{upgrade} after running \mycommand{update} in order to guarantee that the latest version of the packages are installed}.

\begin{command_line}[make]
sudo apt-get upgrade #upgrade all packages to latest versions
\end{command_line}

Note that keeping your system with updated packages is of paramount importance to its security. Any time a vulnerability is found in a package, an arms race starts with the package maintainers who try to update update it with a patch to address this security flaw, and hackers who will try to exploit this flaw. By keeping an old version of the package, your computer will still be vulnerable, even after a patch is already available.


\section{Comparison with .RPM system}

As stated at the beginning of this chapter, there are fundamental differences in how package management is done in different Linux systems. As shown in Section \ref{sec:package_formats}, theer are two main package management systems in Linux. The one based on \mycommand{.DEB} packages, which we covered in this chapter using the \mycommand{apt-get} tool, and one based on .RPM packages, which uses the \mycommand{yum} tool.

In Table \ref{tab:yum_aptget} we present a list of basic \mycommand{yum} commands, together with their \mycommand{apt-get} equivalent commands.

\begin{table}[!htbp]
   \myfloatalign
   \begin{tabularx}{\textwidth}{lXp{45mm}} \toprule
   \tableheadline{.deb} & \tableheadline{.rpm} & \tableheadline{Description}\\ \midrule
   \mycommand{yum install} & \mycommand{apt-get install} & Install a chosen package.\\
   \mycommand{yum remove} & \mycommand{apt-get remove} & Remove a chosen package.\\
   \mycommand{yum check-update} & \mycommand{apt-get update} & Update package file indexes.\\
   \mycommand{yum update} & \mycommand{apt-get update} & Update file indexes and upgrade all packages.\\
   \mycommand{yum clean packages} & \mycommand{apt-get autoremove} & Get rid of no longer needed packages.\\
   \mycommand{yum clean metadata} & \mycommand{apt-get purge PACKAGE}& Get rid of no longer needed config files.\\
   \bottomrule
   \end{tabularx}
\caption{Comparison of \mycommand{yum} and \mycommand{apt-get} commands.}
\label{tab:yum_aptget}
\end{table}



\newpage

\section*{Exercises}
\addcontentsline{toc}{section}{Exercises}

\begin{exercises}
  \item Explain, using your own words, what are the advantages of having a centralized package management system.
  \item Which command can be used to install an application  alled \mycommand{gimp}?
  \item Which command can be used to remove an application called \mycommand{shutter}, while at the same time removing any config files that were installed for this application.
  \item Which command can be used to remove all dependencies that are no longer nedded in your system?
  \item Which command can be used to add a repository with ppa (personal package archives), \mycommand{libreoffice/ppa}, to your repository list.
  \item Are there any security risks with adding third-party repositories to your system? 
  \item Which command can be used to update all packages in your system with their newest version, as shown in the listed repositories?
\end{exercises}